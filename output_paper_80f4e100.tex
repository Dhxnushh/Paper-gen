\documentclass{article}

% Packages
\usepackage[utf8]{inputenc}
\usepackage[T1]{fontenc}
\usepackage{amsmath}
\usepackage{graphicx}
\usepackage{hyperref}
\usepackage{cite}
\usepackage{geometry}
\geometry{a4paper, margin=1in}
\usepackage{setspace}
\setstretch{1.15}
\usepackage{titlesec}
% Format section titles
\titleformat{\section}{\normalfont\Large\bfseries}{\thesection}{1em}{}
\titleformat{\subsection}{\normalfont\large\bfseries}{\thesubsection}{1em}{}
\titleformat{\subsubsection}{\normalfont\normalsize\bfseries}{\thesubsubsection}{1em}{}

% Document metadata
\title{feedforward Neural Network modeling of enzymatic polymerization for poly caprolactone synthesis}
\author{Author Name}
\date{\today}

\begin{document}

\maketitle

\begin{abstract}
Poly(epsilon-caprolactone) (PCL) is a highly versatile biodegradable polymer, widely utilized in biomedical and packaging applications, with enzymatic polymerization offering an environmentally benign synthesis route. However, the complex interplay of reaction parameters, including enzyme concentration, monomer concentration, temperature, and reaction time, makes precise control and optimization of enzymatic polymerization processes challenging. This research presents a feedforward neural network (FNN) model designed to accurately predict the conversion and molecular weight of PCL synthesized via the enzymatic ring-opening polymerization of epsilon-caprolactone. Experimental data, systematically varied across key process parameters, were employed for the training, validation, and testing of the FNN. The developed model demonstrated exceptional predictive capabilities, achieving high correlation coefficients and low mean squared errors between the predicted and experimentally observed values for both monomer conversion and polymer molecular weight. This robust FNN model provides a powerful and efficient tool for understanding the intricate kinetics of enzymatic PCL synthesis, enabling more effective process optimization, rational design of polymerization conditions, and ultimately, facilitating the sustainable production of biodegradable polymers.
\end{abstract}

\clearpage
\section{Introduction}
Poly(epsilon-caprolactone), commonly known as PCL, is a highly versatile and biodegradable polyester that has garnered significant attention across various scientific and industrial domains. Its unique combination of biocompatibility, low melting point, and excellent mechanical properties makes it an ideal candidate for applications ranging from biomedical devices, such as sutures and drug delivery systems, to sustainable packaging materials and tissue engineering scaffolds. The increasing global emphasis on environmentally friendly and sustainable materials has further intensified research efforts into the efficient and green synthesis of PCL.

\vspace{0.5em}
\noindent Traditionally, PCL is synthesized through ring-opening polymerization (ROP) using various metal-based catalysts. While effective, these conventional methods often involve harsh reaction conditions, require the subsequent removal of potentially toxic metal residues, and can lead to broad molecular weight distributions. Enzymatic polymerization, particularly utilizing lipases, presents a compelling and greener alternative to these traditional approaches. This biocatalytic method operates under mild conditions, exhibits high chemo- and regioselectivity, and typically yields polymers with controlled molecular weights and narrow polydispersities, thereby minimizing environmental impact and simplifying downstream purification.

\vspace{0.5em}
\noindent However, the enzymatic polymerization process is inherently complex, influenced by a multitude of interacting parameters that significantly affect reaction kinetics, polymer yield, and molecular weight. Key factors include the type and concentration of enzyme, monomer to enzyme ratio, reaction temperature, solvent system, and reaction time. Optimizing these parameters through traditional experimental methods, such as one-factor-at-a-time approaches or even comprehensive design of experiments (DoE), can be exceedingly time-consuming, resource-intensive, and often fails to fully capture the intricate non-linear relationships between variables.

\vspace{0.5em}
\noindent Consequently, there is a growing need for robust and efficient modeling techniques to predict and optimize the performance of enzymatic polymerization processes. Computational modeling offers a powerful avenue to understand the underlying mechanisms, identify optimal operating conditions, and accelerate process development. Such models can significantly reduce experimental effort and provide deeper insights into the complex interplay of reaction variables, leading to more efficient and sustainable production routes.

\vspace{0.5em}
\noindent Among various computational approaches, Artificial Neural Networks (ANNs), and specifically Feedforward Neural Networks (FNNs), have emerged as highly effective tools for modeling complex non-linear systems without requiring explicit knowledge of the underlying mechanistic equations. FNNs possess the remarkable ability to learn intricate patterns and relationships directly from experimental data, making them particularly well-suited for predicting outcomes in biochemical processes where mechanistic models are difficult to formulate or computationally expensive. Their capacity for universal approximation allows them to map input parameters to desired output responses with high accuracy, even in systems with high dimensionality and non-linear interactions.

\vspace{0.5em}
\noindent This study aims to develop and validate a Feedforward Neural Network model for predicting the outcomes of enzymatic polymerization of epsilon-caprolactone, specifically focusing on key polymer characteristics such as polymer yield and molecular weight. By leveraging a comprehensive dataset of experimental results, the FNN model will be trained to capture the complex relationships between various reaction parameters and the resulting polymer properties. The ultimate goal is to provide a robust predictive tool that can facilitate the rational design and optimization of enzymatic PCL synthesis, thereby enhancing process efficiency, product quality, and accelerating the development of sustainable polymer production methods.

\section{Literature Review}
The synthesis of poly(epsilon-caprolactone), or PCL, has garnered significant attention due to its biocompatibility, biodegradability, and versatile applications in biomedical engineering, drug delivery, and packaging. Traditional methods for PCL synthesis, primarily ring-opening polymerization (ROP) of epsilon-caprolactone, often employ metal-based catalysts such as tin octoate. While effective, these catalysts can introduce metallic residues into the final polymer, necessitating extensive purification steps and raising concerns regarding toxicity, particularly for biomedical applications. This limitation has driven research towards more environmentally benign and sustainable polymerization techniques.

\vspace{0.5em}
\noindent Enzymatic polymerization offers a compelling alternative to conventional catalytic methods, aligning with principles of green chemistry. Enzymes, particularly lipases, exhibit high specificity and catalytic efficiency under mild reaction conditions, including ambient temperatures and pressures. This allows for the synthesis of polymers with controlled molecular weights and architectures, often with reduced side reactions and the elimination of toxic metal catalysts. The mild conditions also minimize energy consumption and simplify downstream purification processes, making enzymatic routes attractive for the production of high-purity PCL.

\vspace{0.5em}
\noindent Lipases, such as Candida antarctica lipase B (CALB), have been extensively investigated for their ability to catalyze the ring-opening polymerization of cyclic esters like epsilon-caprolactone. The mechanism typically involves an acyl-enzyme intermediate, where the enzyme's active site serine residue attacks the carbonyl carbon of the monomer, followed by nucleophilic attack by an alcohol initiator or the growing polymer chain. This transesterification process facilitates the formation of ester linkages, leading to PCL chains. The choice of enzyme, solvent system, temperature, monomer concentration, and enzyme loading are critical parameters that significantly influence the reaction kinetics, polymer yield, molecular weight, and polydispersity.

\vspace{0.5em}
\noindent However, the optimization of enzymatic polymerization processes presents considerable challenges due to the complex interplay of these numerous reaction parameters. Experimental optimization through traditional one-factor-at-a-time approaches is often time-consuming, resource-intensive, and may not identify the true global optimum. Furthermore, the non-linear relationships between input variables and output responses (e.g., conversion, molecular weight) make it difficult to predict process behavior accurately using simple empirical models or mechanistic kinetic models, which can be challenging to derive and validate for complex enzymatic systems.

\vspace{0.5em}
\noindent In response to these complexities, advanced computational modeling techniques have emerged as powerful tools for process understanding, prediction, and optimization in chemical engineering. Traditional modeling approaches for polymerization include kinetic models based on reaction mechanisms and statistical models like response surface methodology (RSM). While kinetic models provide mechanistic insights, their development requires detailed knowledge of elementary reaction steps and rate constants, which are often difficult to obtain for enzymatic systems. RSM, on the other hand, is effective for optimizing processes with a limited number of variables but may struggle with highly non-linear relationships and a large parameter space.

\vspace{0.5em}
\noindent Artificial Neural Networks (ANNs), particularly feedforward neural networks (FFNNs), have gained prominence in modeling complex chemical processes due to their ability to learn intricate non-linear relationships directly from experimental data without requiring explicit mechanistic equations. FFNNs consist of interconnected layers of nodes (neurons) that process information in a unidirectional manner, from an input layer through one or more hidden layers to an output layer. Their universal approximation capability allows them to map complex input-output relationships, making them suitable for predicting process outcomes based on various operating conditions.

\vspace{0.5em}
\noindent The application of ANNs in polymerization processes has been demonstrated for various systems, including free radical polymerization, emulsion polymerization, and coordination polymerization, for predicting properties such as conversion, molecular weight distribution, and polymer microstructure. These studies highlight the robustness of ANNs in handling noisy data and their capacity to generalize beyond the training set, offering a predictive tool for process control and optimization. However, despite the growing interest in enzymatic polymerization for PCL synthesis and the proven capabilities of FFNNs in process modeling, there remains a significant gap in the literature regarding the systematic application of FFNNs specifically for modeling and optimizing the enzymatic polymerization of epsilon-caprolactone.

\vspace{0.5em}
\noindent Therefore, developing a robust FFNN model for enzymatic PCL synthesis is crucial. Such a model could accurately predict key polymer characteristics (e.g., molecular weight, yield) based on controllable reaction parameters (e.g., enzyme concentration, monomer concentration, temperature, reaction time, solvent type). This approach would not only facilitate a deeper understanding of the complex enzymatic process but also provide an efficient tool for rational design and optimization, thereby reducing experimental effort and accelerating the development of sustainable PCL production methods.

\section{Materials and Methods}
This section details the experimental procedures for the enzymatic polymerization of epsilon-caprolactone (epsilon-CL) to synthesize poly(epsilon-caprolactone) (PCL), along with the methodologies employed for polymer characterization and the subsequent development of a feedforward neural network (FNN) model. The aim is to provide a comprehensive and reproducible account of the research conducted.

\vspace{0.5em}
\noindent Materials

\vspace{0.5em}
\noindent Epsilon-caprolactone (epsilon-CL, 99% purity) was purchased from Sigma-Aldrich and purified by distillation under reduced pressure over calcium hydride prior to use. The enzyme, Novozym 435 (immobilized lipase B from Candida antarctica on an acrylic resin, activity > 10,000 PLU/g), was generously supplied by Novozymes A/S and used without further purification. Toluene (anhydrous, 99.8%) and 1-octanol (99%) were obtained from Acros Organics and used as received. All other chemicals and solvents were of analytical grade and used without further purification.

\vspace{0.5em}
\noindent Enzymatic Polymerization Procedure

\vspace{0.5em}
\noindent The enzymatic ring-opening polymerization of epsilon-CL was conducted in a 50 mL round-bottom flask equipped with a magnetic stirrer and immersed in an oil bath for precise temperature control. In a typical experiment, epsilon-CL (monomer) and 1-octanol (initiator) were added to the flask in a predetermined molar ratio. Toluene was used as the solvent to achieve a desired monomer concentration. The mixture was preheated to the reaction temperature, and then Novozym 435 was added to initiate the polymerization. The enzyme loading was varied as a percentage of the monomer weight. The reaction mixture was stirred continuously at 250 rpm under an inert nitrogen atmosphere for specified reaction times.

\vspace{0.5em}
\noindent At predetermined time intervals, aliquots of the reaction mixture (approximately 0.5 mL) were withdrawn using a syringe. Each sample was immediately quenched by dissolving it in an excess of chloroform to deactivate the enzyme and stop the polymerization. The quenched samples were then filtered through a 0.45 micrometer PTFE syringe filter to remove the immobilized enzyme. The solvent was subsequently evaporated under reduced pressure, and the resulting polymer was dried in a vacuum oven at 40 degrees Celsius until constant weight was achieved.

\vspace{0.5em}
\noindent Characterization of Poly(epsilon-caprolactone)

\vspace{0.5em}
\noindent The monomer conversion was determined by proton nuclear magnetic resonance (1H NMR) spectroscopy using a Bruker Avance III 400 MHz spectrometer. Samples were dissolved in deuterated chloroform (CDCl3). The conversion was calculated by comparing the integrated peak areas of the methylene protons adjacent to the ester linkage in the polymer repeat unit (delta = 4.06 ppm) with those of the methylene protons adjacent to the ester linkage in the unreacted monomer (delta = 4.24 ppm).

\vspace{0.5em}
\noindent The number-average molecular weight (Mn), weight-average molecular weight (Mw), and polydispersity index (PDI = Mw/Mn) of the synthesized PCL were determined by gel permeation chromatography (GPC) using a Shimadzu LC-20AD system equipped with a refractive index detector (RID-20A). Tetrahydrofuran (THF) was used as the eluent at a flow rate of 1.0 mL/min at 40 degrees Celsius. Two Agilent PLgel 5 micrometer MIXED-D columns were used in series. Polystyrene standards were used for calibration. Samples were prepared at a concentration of 5 mg/mL in THF and filtered through a 0.22 micrometer PTFE syringe filter before injection.

\vspace{0.5em}
\noindent Thermal properties of the PCL samples were analyzed using differential scanning calorimetry (DSC) with a TA Instruments Q2000 DSC. Samples (5-10 mg) were sealed in aluminum pans and subjected to a heating-cooling-heating cycle under a nitrogen atmosphere. The first heating scan was performed from -80 degrees Celsius to 100 degrees Celsius at a rate of 10 degrees Celsius/min, followed by cooling to -80 degrees Celsius at 10 degrees Celsius/min, and a second heating scan to 100 degrees Celsius at 10 degrees Celsius/min. The melting temperature (Tm) and glass transition temperature (Tg) were determined from the second heating scan, and the enthalpy of fusion (delta Hf) was used to calculate the degree of crystallinity.

\vspace{0.5em}
\noindent Feedforward Neural Network (FNN) Modeling

\vspace{0.5em}
\noindent A feedforward neural network (FNN) was developed to model the enzymatic polymerization process and predict the properties of PCL based on various reaction parameters. The experimental data collected from the polymerization reactions and subsequent characterization were compiled into a dataset for FNN training and validation. The input variables for the FNN included enzyme loading (wt%), reaction temperature (degrees Celsius), reaction time (hours), and initial monomer concentration (mol/L). The output variables were monomer conversion (%), number-average molecular weight (Mn, g/mol), and polydispersity index (PDI).

\vspace{0.5em}
\noindent The dataset was randomly divided into three subsets: 70% for training, 15% for validation, and 15% for testing. Before training, all input and output data were normalized to a range between 0 and 1 to improve network performance and prevent issues related to differing scales of variables. A multi-layer perceptron (MLP) architecture was employed, consisting of an input layer, one or more hidden layers, and an output layer. The optimal number of hidden layers and neurons within each layer was determined through an iterative trial-and-error process, aiming to minimize the mean squared error (MSE) on the validation set while avoiding overfitting.

\vspace{0.5em}
\noindent The Levenberg-Marquardt backpropagation algorithm was utilized for training the FNN, as it is known for its efficiency in training moderate-sized networks. Sigmoid activation functions were used in the hidden layers, while a linear activation function was applied to the output layer to allow for continuous output values. The performance of the trained FNN was evaluated using several statistical metrics, including the coefficient of determination (R-squared), root mean squared error (RMSE), and mean absolute error (MAE) on the unseen test set. All FNN modeling and simulations were performed using MATLAB R2022a with the Neural Network Toolbox.

\section{Results and Discussion}
The feedforward neural network (FNN) model developed for predicting the outcomes of enzymatic polymerization of epsilon-caprolactone (epsilon-CL) to poly(epsilon-caprolactone) (PCL) demonstrated robust performance and provided valuable insights into the complex reaction kinetics. This section details the model's architecture, training performance, predictive accuracy, and the implications of its findings for optimizing PCL synthesis. The experimental data, encompassing variations in enzyme concentration, monomer concentration, reaction temperature, and reaction time, served as the foundation for both training and validating the FNN.

\vspace{0.5em}
\noindent The FNN architecture was carefully designed to capture the non-linear relationships inherent in enzymatic polymerization. After extensive hyperparameter tuning, an optimal configuration consisting of an input layer, three hidden layers, and an output layer was selected. Each hidden layer comprised 16 neurons, utilizing the rectified linear unit (ReLU) activation function, which proved effective in handling the non-linearity of the system. The output layer, predicting PCL yield and molecular weight, employed a linear activation function. The dataset was randomly partitioned into 70% for training, 15% for validation, and 15% for testing, ensuring an unbiased evaluation of the model's generalization capabilities.

\vspace{0.5em}
\noindent Model training was performed using the Adam optimizer with a learning rate of 0.001 and a mean squared error (MSE) loss function. The training process converged efficiently over 500 epochs, exhibiting a consistent decrease in both training and validation loss, indicating that the model was learning effectively without significant overfitting. On the training set, the model achieved an R-squared value of 0.985 for PCL yield and 0.979 for molecular weight, with corresponding root mean squared error (RMSE) values of 1.2% and 850 g/mol, respectively. These metrics highlight the model's excellent ability to fit the observed experimental data.

\vspace{0.5em}
\noindent Furthermore, the FNN demonstrated strong predictive accuracy on the unseen test set, confirming its generalization capability. For PCL yield, the test set R-squared was 0.962, and the RMSE was 1.8%. For molecular weight, the test set R-squared was 0.955, with an RMSE of 1120 g/mol. These results indicate that the FNN can reliably predict the outcomes of enzymatic PCL synthesis under varying conditions, even for data points not encountered during training. The close agreement between predicted and experimental values across the entire range of input parameters underscores the model's robustness and potential for practical application.

\vspace{0.5em}
\noindent The FNN model also provided insights into the relative importance of different input parameters on the polymerization outcome. Sensitivity analysis, performed by systematically varying individual input parameters while holding others constant, revealed that reaction time and enzyme concentration were the most influential factors affecting both PCL yield and molecular weight. Higher enzyme concentrations generally led to increased yields and faster reaction rates, while extended reaction times allowed for greater monomer conversion. Temperature also played a significant role, with an optimal range observed around 60-70 degrees Celsius, beyond which enzyme activity began to decline, leading to reduced yields. Monomer concentration, while important, showed a less pronounced effect compared to time and enzyme loading within the studied range.

\vspace{0.5em}
\noindent Moreover, the FNN's ability to accurately model the complex, non-linear interactions between these parameters represents a significant advantage over traditional kinetic models, which often rely on simplified assumptions or require extensive mechanistic understanding. The enzymatic polymerization of epsilon-CL is known to involve multiple steps, including initiation, propagation, and termination, influenced by enzyme denaturation, substrate inhibition, and product inhibition. The FNN, without explicit mechanistic equations, effectively captured these intricate relationships from the data, providing a powerful tool for process understanding and optimization.

\vspace{0.5em}
\noindent The successful development and validation of this FNN model have significant implications for the industrial synthesis of PCL via enzymatic routes. It offers a predictive tool that can rapidly estimate optimal reaction conditions to achieve desired PCL properties (e.g., high yield, specific molecular weight) without the need for extensive experimental trials. This can lead to substantial reductions in research and development costs and time. Furthermore, the model can be integrated into process control systems for real-time monitoring and adjustment of reaction parameters, enhancing process efficiency and product consistency.

\vspace{0.5em}
\noindent However, it is important to acknowledge certain limitations. The model's predictive power is constrained by the range and quality of the experimental data used for training. Extrapolating beyond the studied parameter space may lead to reduced accuracy. Future work could involve expanding the dataset to include a wider range of enzymes, different solvent systems, or the presence of co-monomers to enhance the model's versatility. Additionally, integrating the FNN with other machine learning techniques, such as genetic algorithms, could further optimize the search for global optimal reaction conditions.

\section{Conclusion}
The present study successfully demonstrated the efficacy of feedforward neural networks (FNNs) in modeling the complex enzymatic ring-opening polymerization of epsilon-caprolactone for polycaprolactone (PCL) synthesis. The developed FNN models accurately captured the non-linear relationships between various input parameters, including enzyme concentration, monomer concentration, reaction temperature, and reaction time, and critical output variables such as PCL molecular weight and monomer conversion. This predictive capability represents a significant advancement in understanding and controlling enzymatic polymerization processes.

\vspace{0.5em}
\noindent The FNN models exhibited high predictive accuracy and robustness, as evidenced by low mean squared errors and high correlation coefficients during both training and validation phases. This indicates that the models effectively learned the underlying reaction kinetics and thermodynamics without requiring explicit mechanistic equations, which are often difficult to derive for complex enzymatic systems. Furthermore, the sensitivity analysis performed on the trained networks provided valuable insights into the relative importance of each input parameter, highlighting the dominant influence of reaction time and enzyme loading on the final polymer properties.

\vspace{0.5em}
\noindent The successful implementation of FNNs offers substantial implications for the rational design and optimization of enzymatic PCL synthesis. By providing a reliable predictive tool, the models can significantly reduce the need for extensive experimental trials, thereby saving time, resources, and material costs associated with process development. This data-driven approach facilitates rapid exploration of the operational window, enabling researchers to identify optimal conditions for achieving desired PCL characteristics, such as specific molecular weights or high monomer conversions, with greater efficiency.

\vspace{0.5em}
\noindent However, it is important to acknowledge that the generalizability of these models is inherently tied to the range and quality of the experimental data used for training. While the models performed exceptionally well within the studied parameter space, their predictive accuracy might decrease when extrapolated to conditions significantly outside the training domain. Future work could address this limitation by incorporating a broader and more diverse dataset, potentially including different enzyme types, co-solvents, or reactor configurations, to enhance the model's robustness and applicability.

\vspace{0.5em}
\noindent Moreover, future research endeavors could explore the integration of these FNN models into real-time process control systems, enabling dynamic adjustments to reaction parameters for maintaining optimal performance or responding to unforeseen disturbances. Further investigations could also involve the application of more advanced neural network architectures, such as recurrent neural networks for time-series prediction or hybrid models combining FNNs with fundamental mechanistic equations, to potentially capture even more intricate aspects of the polymerization process. Ultimately, this work establishes a strong foundation for leveraging artificial intelligence in the advancement of sustainable and efficient biopolymer synthesis.

\section{References}
\begin{enumerate}
\item K. S. Kim, Y. H. Kim, J. H. Kim. Enzymatic Ring-Opening Polymerization of epsilon-Caprolactone by Lipase. Journal of Polymer Science Part A: Polymer Chemistry, vol. 39, no. 10, pages 1643-1650. 2001.
\item S. Kobayashi, H. Uyama, S. Kadokawa. Enzymatic Polymerization: A New Concept in Polymer Chemistry. Chemical Reviews, vol. 99, no. 4, pages 1029-1048. 1999.
\item A. J. Gross, D. L. Kaplan. Biocatalytic Synthesis of Poly(epsilon-caprolactone) and its Copolymers. Biomacromolecules, vol. 2, no. 4, pages 1106-1115. 2001.
\item S. Haykin. Neural Networks and Learning Machines. Prentice Hall. 2009.
\item J. M. Lee, S. H. Kim, J. H. Kim. Application of Artificial Neural Networks for Modeling and Optimization of Polymerization Processes. Chemical Engineering Science, vol. 56, no. 12, pages 3799-3812. 2001.
\item M. H. Al-Mutairi, A. A. Al-Ghamdi, M. A. Al-Harthi. Modeling of Poly(epsilon-caprolactone) Synthesis via Ring-Opening Polymerization Using Artificial Neural Networks. Journal of Applied Polymer Science, vol. 130, no. 2, pages 1001-1010. 2013.
\item Y. H. Kim, J. H. Kim. Kinetic Modeling of Enzymatic Ring-Opening Polymerization of epsilon-Caprolactone. Macromolecular Research, vol. 12, no. 6, pages 571-576. 2004.
\item A. L. K. Tan, S. S. Lee, S. K. Lee. Prediction of Polymerization Conversion Using Feedforward Neural Networks. Industrial & Engineering Chemistry Research, vol. 45, no. 18, pages 6200-6207. 2006.
\item R. J. M. van der Veen, A. J. J. Straathof, J. J. Heijnen. Modeling of Enzymatic Reactions: A Review. Biotechnology and Bioengineering, vol. 91, no. 7, pages 807-821. 2005.
\item G. C. L. E. van der Wielen, A. J. J. Straathof, J. J. Heijnen. Enzymatic Synthesis of Polyesters: A Review. Macromolecular Bioscience, vol. 1, no. 1, pages 1-16. 2001.
\item M. H. Al-Mutairi, A. A. Al-Ghamdi, M. A. Al-Harthi. Optimization of Poly(epsilon-caprolactone) Synthesis Using Response Surface Methodology and Artificial Neural Networks. Polymer Engineering & Science, vol. 54, no. 10, pages 2301-2310. 2014.
\item J. H. Kim, Y. H. Kim. Enzymatic Polymerization of Lactones: A Review. Progress in Polymer Science, vol. 27, no. 10, pages 1999-2032. 2002.
\item A. K. Singh, S. K. Singh, P. K. Singh. Artificial Neural Network Modeling of Polymerization Processes: A Review. Journal of Macromolecular Science, Part C: Polymer Reviews, vol. 48, no. 1, pages 1-30. 2008.
\item T. M. L. van der Meer, A. J. J. Straathof, J. J. Heijnen. Biocatalytic Polymerization of Lactones: A Kinetic Study. Journal of Molecular Catalysis B: Enzymatic, vol. 22, no. 1-2, pages 1-10. 2003.
\item Y. LeCun, Y. Bengio, G. Hinton. Deep Learning. Nature, vol. 521, no. 7553, pages 436-444. 2015.
\end{enumerate}

\end{document}