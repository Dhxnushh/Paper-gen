\documentclass{article}

% Packages
\usepackage[utf8]{inputenc}
\usepackage[T1]{fontenc}
\usepackage{amsmath}
\usepackage{graphicx}
\usepackage{hyperref}
\usepackage{cite}
\usepackage{geometry}
\geometry{a4paper, margin=1in}
\usepackage{setspace}
\setstretch{1.15}
\usepackage{titlesec}
% Format section titles
\titleformat{\section}{\normalfont\Large\bfseries}{\thesection}{1em}{}
\titleformat{\subsection}{\normalfont\large\bfseries}{\thesubsection}{1em}{}
\titleformat{\subsubsection}{\normalfont\normalsize\bfseries}{\thesubsubsection}{1em}{}

% Document metadata
\title{Enzymatic Synthesis of Polycaprolactone using Candida antarctica Lipase B and \\epsilon-Caprolactone}
\author{Researcher Name}
\date{October 19, 2025}

\begin{document}

\maketitle

\begin{abstract}
This is a short abstract about enzymatic polymerization of \\epsilon-caprolactone using Candida antarctica lipase B. The study shows efficient polymerization under mild conditions.
\end{abstract}

\clearpage
\section{Introduction}
Introduction content. This should start on a new page. It describes background and motivation for using CALB for polycaprolactone synthesis.

\section{Methods}
Materials and methods go here.

\section{References}
\begin{enumerate}
\item P. Bellot, T. Chappell, A. Doucet, S. Geva, S. Gurajada, J. Kamps, G. Kazai, M. Koolen, M. Landoni, M. Marx, et al. Report on INEX 2012. In ACM SIGIR Forum, volume 46, pages 50–59. ACM, 2012.
\item C. Bravo-Lillo, L. F. Cranor, J. Downs, S. Komanduri, and M. Sleeper. Improving computer security dialogs. In Human-Computer Interaction–INTERACT 2011, pages 18–35. Springer, 2011.
\item D. P. Coppola. Introduction to international disaster management. Butterworth-Heinemann, 2006.
\item E. Dale and J. S. Chall. A formula for predicting readability. Educational research bulletin, pages 11–28, 1948.
\item O. De Clercq, V. Hoste, B. Desmet, P. Van Oosten, M. De Cock, and L. Macken. Using the crowd for readability prediction. Natural Language Engineering, pages 1–33, 2013.
\end{enumerate}

\end{document}